\chapter{Конструкторская часть}

\section{Проектирование базы данных}

На рисунке \ref{img:db_entities} изображена диаграмма разрабатываемой базы данных в соответствии с ER-диаграммой системы в нотации Чена на рисунке \ref{img:er}.

\imgs{db_entities}{h!}{0.6}{Диаграмма разрабатываемой базы данных}

Таблица Player содержит информацию о футболистоах и имеет следующие поля:

\begin{itemize}
    \item Id -- первичный ключ таблицы (unsigned);
    \item ClubId -- идентификатор клуба (unsigned);
    \item Surname -- фамилия футболиста (string);
    \item Rating -- рейтинг футболиста (unsigned);
    \item Country -- страна футболиста (string);
    \item Price -- цена футболиста (unsigned).
\end{itemize}

\clearpage

Таблица Coach содержит информацию о тренерах и имеет следующие поля:

\begin{itemize}
    \item Id -- первичный ключ таблицы (unsigned);
    \item Surname -- фамилия тренера (string);
    \item Country -- страна тренера (string).
\end{itemize}

Таблица Club содержит информацию о футбольных клубах и имеет следующие поля:

\begin{itemize}
    \item Id -- первичный ключ таблицы (unsigned);
    \item Name -- название футбольного клуба (string);
    \item Country -- страна расположения футбольного клуба (string);
    \item FoundationDate -- год основания футбольного клуба (unsigned).
\end{itemize}

Таблица Squad содержит информацию о составах игроков и имеет следующие поля:

\begin{itemize}
    \item Id -- первичный ключ таблицы (unsigned);
    \item CoachId -- идентификатор тренера (unsigned);
    \item Name -- название состава (string);
    \item Rating -- рейтинг состава (средний рейтинг игроков в составе) (unsigned).
\end{itemize}

Таблица User содержит информацию об игроках и имеет следующие поля:

\begin{itemize}
    \item Id -- первичный ключ таблицы (unsigned);
    \item Login -- логин пользователя (string);
    \item Password -- пароль пользователя (string);
    \item Permission -- права доступа пользователя (string).
\end{itemize}

Таблица SquadPlayer содержит информацию о том, какие игроки в какие клубы были добавлены, имеет следующие поля:

\begin{itemize}
    \item Id -- первичный ключ таблицы (unsigned);
    \item SquadId -- идентификатор состава (unsigned);
    \item PlayerId -- идентификатор футболиста (unsigned).
\end{itemize}

\section{Триггеры базы данных}

При создани базы данных были определены два триггеры. Один срабатывает после добавления в таблицу SquadPlayer новой записи, а другой после удаления старой. Эти два триггера отвечают за обновление рейтинга состава, который определяется средним рейтингом футболистов в нем. Так при добавление игрока в состав вызовется триггер insertRatingTrigger, который вызовет функцию для пересчета рейтинга состава. Аналогично при удаление игрока вызовется триггер deleteRatingTrigger. 

\clearpage

На рисунках \ref{img:updateSquadRating} -- \ref{img:newSquadRating} показаны схемы функций для обновления состава.

\imgs{updateSquadRating}{h!}{0.45}{Схема функции обновления состава}
\imgs{newSquadRating}{h!}{0.45}{Схема функции получения нового рейтинга состава}

\clearpage

\section{Функции базы данных}

Кроме триггеров при создани базы данных была определена функция. Она отвечает за поиск футбольных клубов по заданным параметрам, таким как название футбольного клуба, страна, в которой он располагется, а также минимальная и максимальная даты основания клуба. На рисунке \ref{img:getClubsByParameters} показана схема функции поиска клубов по заданным параметрам.

\imgs{getClubsByParameters}{h!}{0.65}{Схема функции поиска клубов по заданным параметрам}

\clearpage

На рисунках \ref{img:getClubsByName} -- \ref{img:getClubsByCountry} показаны схемы функций поиска клубов по названию и стране.

\imgs{getClubsByName}{h!}{0.47}{Схема функции поиска клубов по имени}
\imgs{getClubsByCountry}{h!}{0.47}{Схема функции поиска клубов по названию страны}

\clearpage

На рисунке \ref{img:getClubsByParameters} показана схема функции поиска клубов по дате основания.

\imgs{getClubsByFoundationDate}{h!}{0.48}{Схема функции поиска клубов по дате основания}

\clearpage

\section{Роли базы данных}

В аналитической части для взаимодействия с Web-приложением было выделено три категории пользователя: гость, авторизованный пользователь и администратор. Для каждой категории требуется создать отдельную роль в базе данных, которые будут называться аналогично.

\begin{enumerate}
    \item Гость (неавторизованный пользователь) сможет просматривать базовую информацию о футболистах, тренерах и клубах, следовательно он должен иметь возможность просматривать таблицы Player, Coach, Club. Также гость сможет зарегистрироваться или авторизоваться на сайте, а значит ему понадобятся права на просмотр и изменение (только на добавление записей) таблиц User, Squad. Прав на добавление информации в другие таблицы, а также на удаление записей из всех таблиц неавторизованный пользовательиметь иметь не будет.
    \item Авторизованный пользователь, помимо привелегий гостя, сможет просматривать информацию о составах других игроков, а также добавлять понравившихся футболистов к себе в состав, следовательно он дополнительно должен иметь права на просмотр таблиц Squad и SquadPlayer, то есть авторизованный пользователь должен иметь права на просмотр всех таблиц в разрабатываемой базе данных. Так как пользователь сможет изменять свой состав, он должен иметь права на добавление и удаление записей из таблицы SquadPlayer. У каждого состава есть свой рейтинг, который отвечает за средний рейтинг игроков в нем. При добавлении или удаление футболиста из него рейтинг должен пересчитываться, следовательно авторизированный пользователь должен иметь права на обновление таблицы Squad.
    \item Администратор имеет все права на все таблицы, так как он должен иметь возможность менять категорию пользователя у других игроков, добавлять новых или удалять старых футболистов из базы данных. Администратору разрешено все.
\end{enumerate}


\section{Вывод}

В данном разделе было произведено проектирование базы данных, в ходе чего была построена диаграмма разрабатываемой базы данных, описаны все поля всех таблиц, а также определены два триггера и одна хранимая функция. На основе категорий пользователя были выделены три ролевые модели на уровне базы данных: гость, авторизованный пользователь и администратор. Подробно были описаны права доступа каждой из них.
