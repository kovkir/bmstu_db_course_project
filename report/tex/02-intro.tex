\maketableofcontents

\intro

В настоящие время огрмное количество людей тратит много времени и денег в компьютерных играх. Одной из самых популярных игр и однозначным лидером в жанре спортивных симуляторов является FIFA 22 --- 29-ая по счёту компьютерная игра из серии FIFA, разработанная компаниями EA Vancouver под издательством Electronic Arts. 

Electronic Arts представила статистику футбольного симулятора FIFA 22 на 22-й день с момента релиза. Жители более 200 стран потратили на FIFA 22 почти 46 трлн минут, что равносильно 88 годам непрерывной игры. За это время пользователи провели более 2 млрд матчей и забили 5 млрд голов. В <<Карьере>> FIFA 22 игроки создали более 3 млн клубов.

По данным издания Eurogamer, за прошлый финансовый год EA заработала на режимах Ultimate Team в FIFA и NHL --- \$ 1,62 млрд. С каждым годом выручка от внутриигровых покупок в спортивных симуляторах только увеличивается. В 2020 году EA заработала \$ 1,491 млрд, а в 2019-м прибыль составила \$ 1,369 млрд \cite{autostat}.

Учитывая все вышеуказанное, можно сделать вывод о том, что интерес к футбольным симуляторам огромный. Создание информационной системы для одного из них поможет пользователям организовать свою команду. Благодаря оригинальным подборкам, игроки смогут подобрать нужного футболиста или тренера.

Целью курсовой работы является разработка базы данных для информационной системы футбольного симулятора. Для достижения поставленной цели необходимо выполнить следующие задачи:

\begin{itemize}
	\item проанализировать существующие решения;
	\item формализовать задачу и определить необходимый функционал;
	\item рассмотреть модели баз данных и выбрать подходящую;
	\item проанализировать существующие СУБД и выбрать нужную;
 	\item спроектировать и разработать БД;
    \item спроектировать и разработать WEB-приложение.
\end{itemize}
