\chapter{Исследовательская часть}

\section{Описание эксперимента}

В качестве эксперимента было решено провести нагрузочное тестирование разработанного Web-приложения с помощью программы Apache JMeter \cite{jmeter}, которая используется для имитации большой нагрузки на сервер, чтобы проверить его на прочность или проанализировать общую производительность при различных типах нагрузки.

В ходе эксперимента в течение 100 секунд к Web-приложению информационной системы футбольного симулятора подключалось 10 пользователей, которые отправляли три вида запросов:

\begin{itemize}
    \item получить список всех футболистов;
    \item получить список всех тренеров;
    \item получить список всех клубов.
\end{itemize}

Суть эксперимента заключалась в том, чтобы сравнить среднее время выполнения этих запросов с учетом увеличения количества активных пользователей. Во время эксперимента таблицы футболистов, тренеров и клубов базы данных содержали 3000, 1000 и 445 строк соответственно.

\clearpage

\section{Результаты эксперимента}

Графики зависимости времени ответа на запросы от времени тестирования и от кол-ва активных пользователей представлены на рисунках \ref{img:load_testing_1} -- \ref{img:load_testing_2}.

\imgs{load_testing_1}{h!}{0.42}{График зависимости времени ответа на запросы от времени тестирования}
\imgs{load_testing_2}{h!}{0.42}{График зависимости времени ответа на запросы от кол-ва активных пользователей}

\clearpage

На рисунке \ref{img:load_testing_table} представлена таблица с результатами эксперимента, столбцы которой отвечают за:

\begin{itemize}
    \item количество разных типов запросов;
    \item среднее время ответа на запрос;
    \item минимальное время ответа на запрос;
    \item максимальное время ответа на запрос.
\end{itemize}

Время в таблице измеряется в миллисекундах.

\imgs{load_testing_table}{h!}{0.65}{Таблица с результатами эксперимента}

\section{Вывод}

В результате эксперимента было установлено, что среднее время запроса на получение списка всех футболистов (0.98 секунды) в 3.82 раза больше, чем время получения списка всех клубов и в 4.11 раз больше времени получения списка всех тренеров. В свое время как время запросов на получения списков тренеров практически не отличается от времени получения списков клубов (быстрее всего в 1.08 раз). 

Стоит отметить, что разработанное Web-приложение плохо себя показало при относительно небольшом кол-ве пользователей. Всего при 10 одновременно активных пользователях, время запроса на получение списка всех футболистов увеличилось до 7 секунд (для клубов и тренеров почти до 2 секунд).
