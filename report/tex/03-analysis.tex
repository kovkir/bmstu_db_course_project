\chapter{Аналитическая часть}

\section{Cуществующие решения}

Так как существует большой интерес к футбольным симуляторам, на рынке уже существуют решения, предоставляющие различный функционал для любителей таких игр.

Рассмотрим только самые популярные из них, такие как:

\begin{itemize}
	\item futbin;
	\item futwiz;
	\item futhead;
	\item fifa companion.
\end{itemize}

Выделим следующие критерии для сравнения выбранных решений:

\begin{enumerate}
	\item возможность поиска игроков по заданным параметрам;
	\item возможность просмотра текущей цены игроков;
	\item возможность собрать собственный состав футболистов;
	\item возможность просмотра рейтинга составов других игроков;
	\item наличие информации об игроках;
	\item наличие информации о тренерах;
	\item наличие информации о клубах.
\end{enumerate}

\clearpage

Результаты сравнения выбранных решений по заданным критериям представлены в таблице \ref{tbl:compare_realizations}.

\captionsetup{justification=raggedleft,singlelinecheck=off}
\begin{table}[H]
    \centering
	\caption{Сравнение существующих решений}
    \label{tbl:compare_realizations}
	\begin{tabular}{|l|l|l|l|l|}
        \hline
        \textbf{Критерий} & \textbf{Futbin} & \textbf{Futwiz} & \textbf{Futhead} & \textbf{FIFA Companion} \\ \hline

        1 & + & + & + & + \\ \hline
        2 & + & + & - & - \\ \hline
        3 & + & + & + & + \\ \hline
        4 & - & - & - & + \\ \hline
        5 & + & + & + & + \\ \hline
        6 & - & - & - & + \\ \hline
        7 & - & - & + & - \\ \hline

    \end{tabular}
\end{table}

Таким образом, ни одно из четырех рассмотренных решений не удовлетворяет всем семи критериям сравнения. Также стоит отметить, что все они являются зарубежными, отечественные аналоги либо отсутствуют, либо слишком непопулярны.

\section{Формализация задачи}

В ходе выполнения курсовой работы необходимо разработать базу данных для хранения информации о футболистах, тренерах, клубах для информационной системы футбольного симулятора. Также необходимо спроектировать и разработать Web-приложение, которое будет предоставлять интерфейс для взаимодействия с базой данных с возможностью создавать каждому отдельному пользователю свой индивидуальный состав футболистов, а также составлять рейтинги по различным параметрам. 

Необходимо предусмотреть возможность добавления новых футболистов в базу данных, а также удаление уже существующих. Требуется реализовать функциональность для разных категорий пользователей, каждый из которых получит свой определённый набор прав.

\clearpage

\section{Формализация данных}

Разрабатываемая база данных для информационной системы футбольного симулятора должна содержать информацию о футболистах, тренерах, клубах, пользователях и их составах. 

\captionsetup{justification=raggedleft,singlelinecheck=off}
\begin{table}[H]
    \centering
	\caption{Категории данных в БД и информация о них}
    \label{tbl:classification_data}
	\begin{tabular}{|l|l|}
        \hline
        \textbf{Категория} & \textbf{Информация} \\ \hline

        Футболист              & Id футболиста, Id клуба, фамилия, рейтинг, \\
                               & страна, цена \\ \hline
        Тренер                 & Id тренера, фамилия, страна \\ \hline
        Клуб                   & Id клуба, название, страна, год основания \\ \hline
        Пользователь           & Id пользователя, логин, пароль \\ 
                               & категория пользователя \\ \hline
        Состав                 & Id состава, Id тренера, название, рейтинг \\ \hline
        Связь футболист-состав & Id связи, Id состава, Id футболиста \\ \hline

    \end{tabular}
\end{table}

Также на рисунке \ref{img:er} изображена ER-диаграмма системы в нотации Чена.

\imgs{er}{h!}{0.3}{ER-диаграмма в нотации Чена}

\clearpage

\section{Формализация категорий пользователя}

Для взаимодействия с Web-приложением было выделено три категории пользователя: гость, авторизованный пользователь и администратор.

Гость (неавторизованный пользователь) сможет просматривать базовую информацию о футболистах, тренерах и клубах. Он сможет воспользоваться подборками для поиска нужных ему игроков. При просмотре списка клубов, гость сможет изучить игроков выбранного клуба.

\imgs{usecase_guest}{h!}{0.5}{Use-case диаграмма для неавторизованного пользователя}

Для получения дополнительного функционала пользователю необходимо зерегистрироваться. Авторизованный пользователь в качестве дополнительного функционала получит возможность просматривать актуальную цену футболистов на рынке, а также организовывать свою команду путем добавления понравившихся игроков и тренера в свой состав. Также он сможет просматривать список составов других игроков.

\imgs{usecase_user}{h!}{0.5}{Use-case диаграмма для авторизованного пользователя}

\clearpage

Администратор --- это авторизованный пользователь, который имеет права на изменение информации в базе данных (он может добавлять или удалять футболистов). Также администратор получит права просматривать список других пользователей Web-сайта, а также изменять их права доступа. В отличие от авторизованного пользователя, который может только просматривать список составов других игроков, администратор имеет права на просмотр футболистов других составов.

\imgs{usecase_admin}{h!}{0.55}{Use-case диаграмма для администратора}

\clearpage

\section{Модели баз данных}

% https://www.techtarget.com/searchdatamanagement/definition/database

База данных --- это любая совокупность информации, которая была систематизирована для быстрого поиска с помощью компьютера. Базы данных используются для хранения, изменения, удаления и доступа к любым данным. Система управления базами данных (СУБД) использует запросы для извлечения информации из базы данных.

Модель базы данных определяет логическую структуру базы данных и то, каким образом данные могут храниться, организовываться и обрабатываться.

% https://www.learntek.org/blog/types-of-databases/

Существует три основных типа модели баз данных:

\begin{itemize}
    \item дореляционные;
    \item реляционные;
    \item постреляционные.
\end{itemize}

\subsection{Дореляционные модели}

К дореляционным моделям баз данных относятся иерархическая и сетевая модели. 

В иерархической модели записи связаны по принципу <<родитель-потомок>>, при этом каждый родитель может быть связан с более чем одним дочерним элементом, но каждый дочерний элемент связан только с одним родителем. У такой модели есть проблемы с отношениями <<многие ко многим>>. Кроме того, в иерархической модели добавление новых связей может привести к массовым изменениям существующей структуры.

% https://mariadb.com/kb/en/understanding-the-hierarchical-database-model/

Сетевая модель базы данных позволяет каждой записи иметь несколько родительских и несколько дочерних записей. Эта модель была создана на замену иерархической модели, так как она может обрабатывать отношения <<многие ко многим>>.

Самая большая проблема с дореляционными моделями баз данных заключалась в отсутствие структурной независимости --- внести структурные изменения в базу данных было очень сложно.

\subsection{Реляционные модели}

В реляционных моделях \cite{db_sql} данные организованы в набор двумерных взаимосвязанных таблиц. Каждая из которых представляет собой набор столбцов и строк, где столбец представляет атрибуты сущности, а строки представляют записи.
Использование таблиц для хранения данных обеспечило простой и эффективный способ хранения структурированной информации, доступа к ней, а также легкую сортировку.

% https://www.scaler.com/topics/dbms/relational-model-in-dbms/

\subsection{Постреляционные модели}

Постреляционные базы данных \cite{db_nosql} реализуют схему с использованием нереляционной модели данных. Такие БД поддерживают логическую модель данных, ориентированную на обработку транзакций. 
Постреляционные базы данных включают хранилища данных с ключевыми значениями, сетевые и графовые базы данных, а также базы данных, ориентированные на документы. 
Недостатком такой модели является сложность решения проблемы обеспечения целостности и непротиворечивости хранимых данных.

\section{Вывод}

В данном разделе были проанализованы аналоги информационной системы футбольного симулятора, ни одно из рассмотренных решений не удовлетворило всем семи выдвинутых критериям сравнения. Также были формализованы поставленная задача, данные, а также категории пользователя. Были рассмотрены модели баз данных. Для решения поставленной задачи была выбрана реляционная модель, так как для разрабатываемой базы данных для информационной системы футбольного симулятора важна целостность хранимых данных, а также простота хранения структурированной информации и ее сортировка.
