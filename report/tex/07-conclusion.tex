\conclusion

Во время выполнения курсовой работы была разработана база данных, а также создано Web-приложение для информационной системы футбольного симулятора. Были рассмотрены модели баз данных и произведен выбор СУБД для реляционной модели БД. После сравнения самых популярных СУБД по выделенным критериям для поставленной задачи был выбран PostgreSQL.

В ходе эксперимента выяснилось, что время запроса на получение списка всех клубов в разы больше, чем время, требуемое на получение списка тренеров (в 4.11 раза) или клубов (в 3.82 раза). Кроме этого эксперимент показал, что Web-приложение плохо себя показывает при относительно небольшом кол-ве активных пользователей (10 человек), так как в этот случае время ожидание на запрос получения всех футболистов возрастает до 7 секунд.

Также в ходе выполнеия курсовой работы были выполнены следующие задачи:

\begin{itemize}
	\item проанализированы существующие решения;
	\item формализована задача и определен необходимый функционал;
	\item рассмотрены модели баз данных и выбрана подходящая;
	\item проанализированы существующие СУБД и выбрана нужная;
 	\item спроектирована и разработана БД;
    \item спроектировано и разработано WEB-приложение.
\end{itemize}

Поставленная цель была достигнута.
