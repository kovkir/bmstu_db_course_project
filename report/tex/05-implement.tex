\chapter{Технологическая часть}

\section{Средства реализации ПО}

Для написания Web-приложения был выбран язык программирования C\# \cite{csharp}. Такой решение обусловлено следующими причинами.

\begin{itemize}
    \item Главной причиной выбора C\# стала его Web-поддержка.
    \item Данный язык поддерживает объектно-ориентированную парадигму программирования, что было обязательным требованием для выбора языка разработки.
    \item C\# многим похож на ранее знакомые мне языки программирования, такие как C++ и C, что также помогло склонить чашу весов в пользу C\#. 
\end{itemize}

В качестве среды разработки был выбран Visual Studio \cite{vs}. Такой решение обусловлено следующими причинами.

\begin{itemize}
	\item Главным требованием при выборе среды разработки была Web-поддержка, которая имеется у Visual Studio.
	\item VS является бесплатным программным обеспечением, что также было важным требованием при выборе среды.
	\item Visual Studio доступна на macOS.
\end{itemize}

При выборе платформы Web-разработки рассматривались два фреймворка на платформе .NET. Это ASP.NET \cite{asp_net} и ASP.NET Core \cite{asp_net_core}. В пользу выбора второго из них повлияли следующие причины.

\begin{itemize}
    \item Данная платформа имеет большую производительность по сравнению с ASP.NET.
    \item ASP.NET Core предназначена для Windows, macOS и Linux, в свою очередь как ASP.NET только для Windows.
\end{itemize}

Другие платформы не рассматривались, так как ранее в качестве языка программирования был выбран C\#, а средой разработки Visual Studio.

\section{Архитектура приложения}

За основу архитектуры разрабатываемого Web-приложения была взята модель MVC (MODEL, VIEW, CONTROLLER). Эта модель помогает обеспечить разделение между бизнес-логикой программного обеспечения и отображением. 

Фреймворк MVC включает в себя следующие три компонента.

\begin{itemize}
    \item Модель -- управляет данными и бизнес-логикой.
    \item Представление --  используется для всей логики пользовательского интерфейса приложения.
    \item Контроллер -- обеспечивает взаимосвязь между представлениями и моделью, поэтому он действует как посредник.
\end{itemize}

ASP.NET Core поддерживает модель разработки MVC. ASP.NET Core MVC framework \cite{mvc} -- это легкий, легко тестируемый фреймворк представления, который интегрирован с существующими ASP.NET Core функциями.

\section{Выбор СУБД}

В аналитическом разделе при выборе типа модели баз данных была выбрана реляционная модель, следовательно выбор СУБД будет производиться для реляционной модели баз данных. Рассмотрим только самые популярные из них: MySQL, Oracle, PostgreSQL \cite{postgresql}, Microsoft SQL Server.

Выделим слудующие критерии для сравнения выбранных СУБД:

\begin{enumerate}
    \item бесплатное распространение СУБД, что будет главным требованием при выборе;
    \item наличие подробной документации, которая поможет разобраться во всех возникших проблемах;
    \item наличие высокого уровня оптимизации СУБД, что будет отличным плюсом при выборе;
    \item опыт работы с СУБД, не является основным требованием, но очень желательным.
\end{enumerate}

Результаты сравнения выбранных СУБД по заданным критериям представлены в таблице \ref{tbl:compare_DBMS}.

\captionsetup{justification=raggedleft,singlelinecheck=off}
\begin{table}[H]
    \centering
	\caption{Сравнение выбранных СУБД}
    \label{tbl:compare_DBMS}
	\begin{tabular}{|l|l|l|l|l|}
        \hline
        \textbf{Критерий} & \textbf{MySQL} & \textbf{Oracle} & \textbf{PostgreSQL} & \textbf{Microsoft SQL Server} \\ \hline

        1 & + & - & + & - \\ \hline
        2 & + & + & + & + \\ \hline
        3 & + & - & + & - \\ \hline
        4 & - & - & + & - \\ \hline

    \end{tabular}
\end{table}

По результатам сравнения в качестве СУБД для реляционной модели баз данных был выбран PostgreSQL, так как он едениственный удовлетворил всем четырем критериям сравнения.

\section{Реализация триггеров}

Триггеры обновления рейтинга состава после добавления или удаления футболиста представлены в листинге \ref{lst:trigger}.

\mylisting[sql]{trigger.sql}{firstline=1,lastline=52}{Триггеры обновления рейтинга состава}{trigger}{}

\section{Реализация функции}

Функция поиска клубов по заданным параметрам представлены в листинге \ref{lst:function}.

\mylisting[sql]{function.sql}{firstline=1,lastline=112}{Функция поиска клубов по заданным параметрам}{function}{}

\section{Реализация ролевых моделей}

При создании базы данных были определены три роли: гость, авторизованный пользователь и администратор. Создание ролей, а также наделение их правами представлена в листинге \ref{lst:roles}.

\mylisting[sql]{roles.sql}{firstline=1,lastline=79}{Роли базы данных}{roles}{}

\clearpage

\section{Демонстрация работы Web-приложения}

\subsection{Демонстрация возможностей гося}

Когда гость (неавторизованный пользователь) попадает на Web-сайт, его встречает домашняя страница (рисунки \ref{img:home_guest_1} -- \ref{img:home_guest_2}), на которой он может сделать поиск футболистов, тренеров или клубов по параметрам.

\imgs{home_guest_1}{h!}{0.25}{Поиск футболистов по параметрам (для гостя)}
\imgs{home_guest_2}{h!}{0.25}{Поиск тренеров и клубов по параметрам (для гостя)}

\clearpage

Также гость может просматривать базовую информацию о футболистах (рисунок \ref{img:players_guest}), тренерах (рисунок \ref{img:coaches_guest}) и клубах (рисунок \ref{img:clubs_guest}).

\imgs{players_guest}{h!}{0.25}{Список всех футболистов (для гостя)}
\imgs{coaches_guest}{h!}{0.25}{Список всех тренеров (для гостя)}

\clearpage

При просмотре списка клубов, гость может изучить игроков выбранного клуба (рисунок \ref{img:club_players_guest}).

\imgs{clubs_guest}{h!}{0.25}{Список всех клубов (для гостя)}
\imgs{club_players_guest}{h!}{0.25}{Список футболистов AS Monaco (для гостя)}

\clearpage

Также гость может авторизоваться (рисунок \ref{img:login}) или зарегистрироваться (рисунок \ref{img:register}).

\imgs{login}{h!}{0.25}{Авторизация пользователя}
\imgs{register}{h!}{0.25}{Регистрация пользователя}

\clearpage

\subsection{Демонстрация возможностей авторизованного пользователя}

Авторизованный пользователь при поиске футболистов по параметрам сможет выбирать, где их искать (в клубе или среди всех футболистов), также он сможет просматривать список составов других игроков (рисунки \ref{img:home_user} -- \ref{img:squads_user}).

\imgs{home_user}{h!}{0.25}{Поиск футболистов по параметрам (для пользователя)}
\imgs{squads_user}{h!}{0.25}{Список составов других игроков (для пользователя)}

\clearpage

Авторизованный пользователь в качестве дополнительного функционала получает возможность просматривать актуальную цену футболистов на рынке, а также организовывать свою команду путем добавления понравившихся игроков и тренера в свой состав. Списки всех футболистов и тренеров с новыми возможностями прдставлены на рисунках \ref{img:players_user} -- \ref{img:coaches_user}.

\imgs{players_user}{h!}{0.25}{Список всех футболистов (для пользователя)}
\imgs{coaches_user}{h!}{0.25}{Список всех тренеров (для пользователя)}

\clearpage

На рисунках \ref{img:player_added} -- \ref{img:player_deleted} показано добавление и удаление футболистов из состава. Добавление и удаление тренера происходит аналогично.

\imgs{player_added}{h!}{0.25}{Добавление футболиста в состав}
\imgs{player_deleted}{h!}{0.25}{Удаление футболиста из состава}

\clearpage

Дабавленных в состав футболистов можно отсортировать по фамилии, рейтингу, стране, названию команды или цене (как по возрастанию, так и по убыванию). В качестве примера на рисунках \ref{img:sort_players_by_surname} -- \ref{img:sort_players_by_price} показан состав с отсортированными по фамилии в алфавитном порядке футболистами, а также по убывнию их цены.

\imgs{sort_players_by_surname}{h!}{0.25}{Футболисты, отсортированными по фамилии в алфавитном порядке}
\imgs{sort_players_by_price}{h!}{0.25}{Футболисты, отсортированными по убывнию их цены}

\subsection{Демонстрация возможностей администратора}

На домашней странице администратор получил возможность добавлять или удалять футболистов из базы данных (рисунок \ref{img:add_delete_players}). Он также может просматривать список зарегистрированных пользователей и изменять их права доступа (рисунок \ref{img:users_admin}).

\imgs{add_delete_players}{h!}{0.25}{Поля для добавления или удаления футболистов из БД}
\imgs{users_admin}{h!}{0.25}{Список зарегистрированных пользователей}

\clearpage

В отличие от авторизованного пользователя, который может только просматривать список составов других игроков, администратор имеет права на просмотр футболистов других клубов (рисунки \ref{img:squads_admin} -- \ref{img:squad_players}).

\imgs{squads_admin}{h!}{0.25}{Список составов других игроков (для администратора)}
\imgs{squad_players}{h!}{0.25}{Футболисты состава Pink Rabbit}

\clearpage

\section{Вывод}

В данном разделе были описаны средства реализации ПО, архитектура приложения, был произведен выбор СУБД для реляционной модели БД. После сравнения самых популярных СУБД по выделенным критериям был выбран PostgreSQL, так как он единственный подошел по всем требованиям. Также были описаны реализации триггеров, функций, а также ролевые модели базы данных. Была продемонстрирована работа Web-приложения для каждой категории пользователя.
